\documentclass{article}
\usepackage[T2A]{fontenc}
\usepackage[utf8]{inputenc}
\usepackage{amsthm}
\usepackage{amsmath}
\usepackage{amssymb}
\usepackage{amsfonts}
\usepackage{mathrsfs}
\usepackage[12pt]{extsizes}
\usepackage{fancyvrb}
\usepackage{indentfirst}
\usepackage[
  left=2cm, right=2cm, top=2cm, bottom=2cm, headsep=0.2cm, footskip=0.6cm, bindingoffset=0cm
]{geometry}
\usepackage[english,russian]{babel}


\begin{document}
\section*{Вариант 29}
Введем функцию

\begin{equation}
  \Phi(s^*(i), l) = (s_i^*(i) - s_i^{\circ} - 1)^2 + (s_i^*(i) - s_i^{\circ} + 1)^2 + \sum_{j \in \Omega_i \\ j \neq l} (s_j^*(i) - s_j^{\circ})^2, \quad i \in I, \; l \in \Omega_i,
\end{equation}
которую назовем функцией выбора. Тогда положим 

\begin{equation}
  \theta_{ij}(s^*(i)) = 
\begin{cases}
1, & \text{если } j = \arg\min\limits_{t \in \Omega_i} \Phi(s^*(i), t); \\
0, & \text{в противном случае.}
\end{cases}
\end{equation}
При таком способе формирования вероятностей перехода требований обеспечивается переход
сети за счет перехода требования из системы $S_i$ в систему $S_j$ из локального состояния $s^*(i)$
в смежное локальное состояние, имеющее наибольший суммарный потенциал, определяемый
выражением

\begin{equation*}
  V(X(s^*(i),j)) = \sum_{s \in X(s^*(i),j)} V(s).
\end{equation*}
\end{document}

